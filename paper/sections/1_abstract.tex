\begin{center}
    \Large{\textbf{Аннотация}}
\end{center}

    В данной работе рассматривается проблема оптимизации "черного ящика". В такой постановке задачи не имеется доступа к градиенту целевой функции, поэтому его необходимо как-то оценить. Предлагается новый способ аппроксимации \texttt{JAGUAR}, который запоминает информацию из предыдущих итераций и требует $\mathcal{O}(1)$ обращений к оракулу. Я реализую эту аппроксимацию для алгоритма Франка-Вольфа и докажу сходимость для выпуклой постановки задачи. Также в данной работе рассматривается стохастическая задача минимизации на множестве $Q$ с шумом в оракуле нулевого порядка, такая постановка довольно непопулярна в литературе, но мы доказали, что \texttt{JAGUAR}-аппроксимация является робастной не только в детерминированных задачах минимизации, но и в стохастическом случае. Я провел эксперименты по сравнению моего градиентного оценщика с уже известными в литературе и подтверждаю доминирование своего метода.